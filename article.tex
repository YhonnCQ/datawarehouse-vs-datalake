\documentclass[twoside,twocolumn]{article}

\usepackage{blindtext} 
\usepackage{graphicx}
\usepackage[sc]{mathpazo} 
\usepackage[T1]{fontenc} 
\linespread{1.05} 
\usepackage{microtype} 


\usepackage[spanish,english]{babel} 


\usepackage[hmarginratio=1:1,top=32mm,columnsep=20pt]{geometry} 
\usepackage[hang, small,labelfont=bf,up,textfont=it,up]{caption} 
\usepackage{booktabs} 


\usepackage{lettrine} 


\usepackage{enumitem} 
\setlist[itemize]{noitemsep} 


\usepackage{abstract} 
\renewcommand{\abstractnamefont}{\normalfont\bfseries} 
\renewcommand{\abstracttextfont}{\normalfont\small\itshape} 


\usepackage{titlesec} 
\renewcommand\thesection{\Roman{section}} % 
\renewcommand\thesubsection{\roman{subsection}} 
\titleformat{\section}[block]{\large\scshape\centering}{\thesection.}{1em}{} 
\titleformat{\subsection}[block]{\large}{\thesubsection.}{1em}{} 


\usepackage{fancyhdr} 
\pagestyle{fancy} 
\fancyhead{} 
\fancyfoot{} 
\fancyhead[C]{Comparative Datawarehouse vs Datalake - \today} 
\fancyfoot[RO,LE]{\thepage} 


\usepackage{titling} 

%----------------------------------------------------------------------------------------
%	TILULOS
%----------------------------------------------------------------------------------------


\setlength{\droptitle}{-4\baselineskip} 

\pretitle{\begin{center}\Huge\bfseries} 
\posttitle{\end{center}} 
\title{Comparative Datawarehouse vs Datalake} 
\author{
	Valdivia Guzman, Alejandra Maria\\
	\and
	Pazos Alarcón, Christian Joshua\\
	\and
	Farfan Colque, Mathius Omar\\
	\and
	Condori Quispe, Yhónn Joel\\
}
\date{\today} 
\renewcommand{\maketitlehookd}{
\selectlanguage{spanish} 
\begin{abstract}
\noindent 
Cada día se generan enormes cantidades de datos procedentes de las tecnologías digitales y los sistemas de
información. Por ello, el tratamiento de estos datos masivos requiere una arquitectura específica y un buen
conocimiento de cómo manejar los datos. Los sistemas tradicionales de gestión de bases de datos ya no pueden
utilizarse para este tipo de datos, ya que fueron diseñados originalmente para datos limitados y estructurados. Por
otra parte, se ha desarrollado una arquitectura específica conocida como Data Lake con el fin de extraer información
valiosa oculta en los datos. El objetivo principal de este artículo es explorar las dos arquitecturas, a saber, el almacén
de datos y el lago de datos. Además, describe las principales diferencias y expone los factores clave de cada una.
\end{abstract}
\selectlanguage{english} 
\begin{abstract}
\noindent 
Each day huge quantities of data are generated from digital technologies and information systems. Therefore,
processing these massive data requires a specific architecture and a good knowledge on how to handle data.
Traditional databases management system can no longer be used for this type of data since they were originally
designed for limited and structured data. Moreover, dedicated architecture known as Data Lake has been developed
in order to extract valuable information hidden in data. The main objective of this paper is to explore the two
architectures, namely, data warehouse and data lake. Furthermore, it describes the main differences and exposes
key factors of each one. 
\end{abstract}
}

%----------------------------------------------------------------------------------------

\begin{document}

% Print the title
\maketitle

%----------------------------------------------------------------------------------------
%	Introduction
%----------------------------------------------------------------------------------------

\section{Introduction}

\lettrine[nindent=0em,lines=3]{R}elational Databases or RDBMS played a key role in making data management
mainstream. They are good with highly structured, low quantity data of the pre-internet era. The advent of the
internet coincided with the ambitions of large organizations to incorporate a 360-degree view of their customer
database. This led to a new type of storage destination known as Data Warehouses. Today, even this storage
destination has a smaller subset, popularly known as Data Lakes. When it comes to the field of Database
Storage, the Data Warehouse vs Data Lake question is a relatively tough choice.\\

Even today very few people understand the differences between these 2 types of storage.  Although Data
Warehouses are good at handling structured Big Data, companies quickly realized that the Data Warehouses
might not cater to the rising demand for insights into unstructured data. The 21st century witnessed a deluge
of data collection by organizations not only from internal sources but also from public repositories. They
needed a technology that could complement the capabilities of Data Warehouses, an extension that can
facilitate the immense unstructured aspect of Big Data. As a result, Data Lakes came into existence.
In this article we are going to explore both, Data Lakes and Warehouses, unfold their key differences and discuss
their usage in the context of an organization.


%----------------------------------------------------------------------------------------
%	State of Art
%----------------------------------------------------------------------------------------

\section{State of Art}

\subsection{Data Warehouse}

Data Warehouse is a set of data produced for decision making, where current and historical data of potential
usefulness for decision making by managers throughout the organization is stored. The data is structured and
available in a form that allows analytical processing activities: OLAP, data mining, querying, reporting and other
DSS applications. In exact terms Data Warehouse is defined as a collection of data, subject-oriented, integrated,
time-specific and non-volatile information, to enable the decision making process by
management\cite{gonzales2012impacto}.\\

The Data Warehouse is more than the consolidation of all the company's operational databases, as it takes into
account business intelligence, external data and data associated with specific dates, making it a unique type of
database. An important aspect of the Data Warehouse is that it is more of an architecture than a technology, and
although there is a relationship between Data Warehousing and database technology, they are not the same, and
Data Warehousing requires the support of several different types of technology\cite{gonzales2012impacto}.

\subsection{Data Warehouse Components}

\begin{center}
	\includegraphics[width=7cm]{./images/dw-components}
	Data Warehouse Architecture\cite{Sharma2021}.
\end{center}

\subsubsection{Operational Source}
\begin{itemize}	
	
	\item An operational Source is a data source consists of Operational Data and External Data.
	\item Data can come from Relational DBMS like Informix, Oracle\cite{Sharma2021}.
	
\end{itemize}

\subsubsection{Load Manager}
\begin{itemize}	
	
	\item The Load Manager performs all operations associated with the extraction of loading data in the data warehouse.
	\item These tasks include the simple transformation of data to prepare data for entry into the warehouse\cite{Sharma2021}.
	
\end{itemize}

\subsubsection{Warehouse Manage}
\begin{itemize}	
	
	\item The warehouse manager is responsible for the warehouse management process.
	\item The operations performed by the warehouse manager are the analysis, aggregation, backup and
	collection of data, de-normalization of the data\cite{Sharma2021}.
	
\end{itemize}

\subsubsection{Query Manager}
\begin{itemize}	
	
	\item Query Manager performs all the tasks associated with the management of user queries.
	\item The complexity of the query manager is determined by the end-user access operations tool and the
	features provided by the database\cite{Sharma2021}.
	
\end{itemize}

\subsubsection{Detailed Data}
\begin{itemize}	
	
	\item It is used to store all the detailed data in the database schema.
	\item Detailed data is loaded into the data warehouse to complement the data collected\cite{Sharma2021}.
	
\end{itemize}

\subsubsection{Summarized Data}
\begin{itemize}	
	
	\item Summarized Data is a part of the data warehouse that stores predefined aggregations.
	\item These aggregations are generated by the warehouse manager\cite{Sharma2021}.
	
\end{itemize}

\subsubsection{Archive and Backup Data}
\begin{itemize}	
	
	\item The Detailed and Summarized Data are stored for the purpose of archiving and backup.
	\item The data is relocated to storage archives such as magnetic tapes or optical disks\cite{Sharma2021}.
	
\end{itemize}

\subsubsection{Metadata}
\begin{itemize}	
	
	\item Metadata is basically data stored above data.
	\item It is used for extraction and loading process, warehouse, management process, and query management
	process\cite{Sharma2021}.
	
\end{itemize}

\subsubsection{End User Access Tools}
\begin{itemize}	
	
	\item End-User Access Tools consist of Analysis, Reporting, and mining.
	\item By using end-user access tools users can link with the warehouse\cite{Sharma2021}.
	
\end{itemize}

\subsection{Data Warehouse Examples}
In general, DWH is implemented in companies that handle large volumes of data related to customers, products
or transactions. Among the sectors that make use of this tool, the following can be mentioned:

\subsubsection{DWH in the telecom sector}
The world of telecommunications is extremely dynamic and competitive. For this reason, organizations resort to
tools that allow them to study their internal productivity, the market, its changes and behaviors in the face of
new technologies.\\

Therefore, telecommunications companies use data warehouses to store the data of millions of customers. This
involves the backup of invoices, services used, records of calls made, equipment sold, among others. All this
information is very useful for activities such as:

\begin{itemize}	
	
	\item The design of marketing strategies
	\item Audits in the operations area
	\item Service delivery analysis
	\item Forecasts of risks of customer leakage and others
	
\end{itemize}

\subsubsection{DWH in the mass consumer sector}
Companies implement data warehousing to stay competitive in the market. In this way they can predict, for
example, the amount of production they will need to meet demand in a given time range.\\

Retail chains can also share certain accesses to their data warehouses with their suppliers. This will give
manufacturers information related to the supply of products and their sale to the end consumer.\\

This whole process allows coordinating management between producers and stores, in addition to accessing
data that is decisive for the development of marketing campaigns.

\subsubsection{DWH in the transport sector}
In both the travel and distribution sectors, the use of DWH is an excellent tool for storing customer
information, most frequented destinations, freight management, luggage tracking, among others.\\

Thus, data such as travel reservations to a certain destination, or delivery times of orders, will allow the
development of analysis for the creation of promotions or for diagnostics of the organization's logistics processes.


\subsection{Datalake}
A data lake is a scalable storage and analysis system for data of any type, retained in their native format and used
mainly by data specialists (statisticians, data scientists or analysts) for knowledge extraction. Its characteristics include\cite{sawadogo2021data}:

\begin{itemize}	
	
	\item a metadata catalog that enforces data quality.
	\item data governance policies and tools.
	\item accessibility to various kinds of users.
	\item integration of any type of data.
	\item a logical and physical organization.
	\item scalability in terms of storage and processing.
	
\end{itemize}

\subsection{Datalake Components}

\subsubsection{Data Ingestion}
Temporary loading layer in which data passes through basic checks before being stored in the raw data layer. It can
perform\cite{Dertiano2021}:

\begin{itemize}	
	
	\item Basic quality controls, such as possible filters according to the origin of the data, discarding unknown sources.
	\item Data encryption processes if required for security reasons.
	\item Simple metadata and traceability records by tags, storing the origin of the data, date and time of loading, format and
	other technical characteristics, privacy and security level, encryption algorithm, etc.

\end{itemize}

\subsubsection{Data Storage}
Layer without established schema where all data, structured or unstructured, are stored without undergoing adaptations. It is a
layer that requires expert data discovery analysts using big data tools (Hive, Spark, Map Reduce, etc.)\cite{Dertiano2021}.

\subsubsection{Data Processing}
Once the data analysts have performed data discovery on the raw data, it may be necessary to process and adapt certain
datasets to accommodate them in a layer of recurrent use. Advanced data quality, integrity and other adaptations can take
place in this layer to provide a trusted layer of data exploration that can be accessed by other users\cite{Dertiano2021}.

\subsubsection{Data Access}
This is a more advanced layer where, finally, data is made available to business analysts. These analysts will be able to generate
reports and analysis to answer business questions and support decision making\cite{Dertiano2021}.

\subsection{Datalake Examples}

\subsubsection{Victoria University}
Victoria University, an Australian public university with more than 40,000 students, is using Cazena to collect and manage
the massive amounts of data generated by student interactions and their business systems. With Cazena’s Instant Data Lake
with Cloudera on AWS, VU has improved educational outcomes and business operations, helping to attract and retain
students\cite{Victoria2021}.

\begin{center}
	\includegraphics[width=7cm]{./images/dl-example}
\end{center}

%----------------------------------------------------------------------------------------
%	Conclusions
%----------------------------------------------------------------------------------------

\section{Conclusions}
\begin{itemize}	
	
	\item There’s no better way to choose which data storage platform best fits your company than to evaluate it based on
	your needs and business operation.
	\item Furthermore, data lakes and data warehouses are two inseparable components that are extremely effective when
	both are utilized well.
	\item Not to mention, data lakes are becoming more and more user-friendly while data warehouses continue to prove
	their worth in terms of data analysis and reporting.

\end{itemize}

%----------------------------------------------------------------------------------------
%	References
%----------------------------------------------------------------------------------------

\bibliographystyle{plain} 
\bibliography{references} 
\end{document}