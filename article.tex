\documentclass[twoside,twocolumn]{article}

\usepackage{blindtext} 
\usepackage{graphicx}
\usepackage[sc]{mathpazo} 
\usepackage[T1]{fontenc} 
\linespread{1.05} 
\usepackage{microtype} 


\usepackage[spanish,english]{babel} 


\usepackage[hmarginratio=1:1,top=32mm,columnsep=20pt]{geometry} 
\usepackage[hang, small,labelfont=bf,up,textfont=it,up]{caption} 
\usepackage{booktabs} 


\usepackage{lettrine} 


\usepackage{enumitem} 
\setlist[itemize]{noitemsep} 


\usepackage{abstract} 
\renewcommand{\abstractnamefont}{\normalfont\bfseries} 
\renewcommand{\abstracttextfont}{\normalfont\small\itshape} 


\usepackage{titlesec} 
\renewcommand\thesection{\Roman{section}} % 
\renewcommand\thesubsection{\roman{subsection}} 
\titleformat{\section}[block]{\large\scshape\centering}{\thesection.}{1em}{} 
\titleformat{\subsection}[block]{\large}{\thesubsection.}{1em}{} 


\usepackage{fancyhdr} 
\pagestyle{fancy} 
\fancyhead{} 
\fancyfoot{} 
\fancyhead[C]{Comparative Datawarehouse vs Datalake - \today} 
\fancyfoot[RO,LE]{\thepage} 


\usepackage{titling} 

%----------------------------------------------------------------------------------------
%	TILULOS
%----------------------------------------------------------------------------------------


\setlength{\droptitle}{-4\baselineskip} 

\pretitle{\begin{center}\Huge\bfseries} 
\posttitle{\end{center}} 
\title{Comparative Datawarehouse vs Datalake} 
\author{
	Valdivia Guzman, Alejandra Maria\\
	\and
	Pazos Alarcón, Christian Joshua\\
	\and
	Farfan Colque, Mathius Omar\\
	\and
	Condori Quispe, Yhónn Joel\\
}
\date{\today} 
\renewcommand{\maketitlehookd}{
\selectlanguage{spanish} 
\begin{abstract}
\noindent 
Lorem ipsum dolor sit amet, consectetur adipiscing elit. Morbi vulputate tempus molestie. 
\end{abstract}
\selectlanguage{english} 
\begin{abstract}
\noindent 
Lorem ipsum dolor sit amet, consectetur adipiscing elit. Morbi vulputate tempus molestie. 
\end{abstract}
}

%----------------------------------------------------------------------------------------

\begin{document}

% Print the title
\maketitle

%----------------------------------------------------------------------------------------
%	Introduction
%----------------------------------------------------------------------------------------

\section{Introduction}

\lettrine[nindent=0em,lines=3]{L}orem ipsum dolor sit amet, consectetur adipiscing elit.
Morbi vulputate tempus molestie.

%----------------------------------------------------------------------------------------
%	State of Art
%----------------------------------------------------------------------------------------

\section{State of Art}

\subsection{Data Warehouse}

Data Warehouse is a set of data produced for decision making, where current and historical data of potential
usefulness for decision making by managers throughout the organization is stored. The data is structured and
available in a form that allows analytical processing activities: OLAP, data mining, querying, reporting and other
DSS applications. In exact terms Data Warehouse is defined as a collection of data, subject-oriented, integrated,
time-specific and non-volatile information, to enable the decision making process by
management\cite{gonzales2012impacto}.\\

The Data Warehouse is more than the consolidation of all the company's operational databases, as it takes into
account business intelligence, external data and data associated with specific dates, making it a unique type of
database. An important aspect of the Data Warehouse is that it is more of an architecture than a technology, and
although there is a relationship between Data Warehousing and database technology, they are not the same, and
Data Warehousing requires the support of several different types of technology\cite{gonzales2012impacto}.

\subsection{Data Warehouse Components}

\begin{center}
	\includegraphics[width=7cm]{./images/dw-components}
	Data Warehouse Architecture\cite{Sharma2021}.
\end{center}

\subsubsection{Operational Source}
\begin{itemize}	
	
	\item An operational Source is a data source consists of Operational Data and External Data.
	\item Data can come from Relational DBMS like Informix, Oracle\cite{Sharma2021}.
	
\end{itemize}

\subsubsection{Load Manager}
\begin{itemize}	
	
	\item The Load Manager performs all operations associated with the extraction of loading data in the data warehouse.
	\item These tasks include the simple transformation of data to prepare data for entry into the warehouse\cite{Sharma2021}.
	
\end{itemize}

\subsubsection{Warehouse Manage}
\begin{itemize}	
	
	\item The warehouse manager is responsible for the warehouse management process.
	\item The operations performed by the warehouse manager are the analysis, aggregation, backup and
	collection of data, de-normalization of the data\cite{Sharma2021}.
	
\end{itemize}

\subsubsection{Query Manager}
\begin{itemize}	
	
	\item Query Manager performs all the tasks associated with the management of user queries.
	\item The complexity of the query manager is determined by the end-user access operations tool and the
	features provided by the database\cite{Sharma2021}.
	
\end{itemize}

\subsubsection{Detailed Data}
\begin{itemize}	
	
	\item It is used to store all the detailed data in the database schema.
	\item Detailed data is loaded into the data warehouse to complement the data collected\cite{Sharma2021}.
	
\end{itemize}

\subsubsection{Summarized Data}
\begin{itemize}	
	
	\item Summarized Data is a part of the data warehouse that stores predefined aggregations.
	\item These aggregations are generated by the warehouse manager\cite{Sharma2021}.
	
\end{itemize}

\subsubsection{Archive and Backup Data}
\begin{itemize}	
	
	\item The Detailed and Summarized Data are stored for the purpose of archiving and backup.
	\item The data is relocated to storage archives such as magnetic tapes or optical disks\cite{Sharma2021}.
	
\end{itemize}

\subsubsection{Metadata}
\begin{itemize}	
	
	\item Metadata is basically data stored above data.
	\item It is used for extraction and loading process, warehouse, management process, and query management
	process\cite{Sharma2021}.
	
\end{itemize}

\subsubsection{End User Access Tools}
\begin{itemize}	
	
	\item End-User Access Tools consist of Analysis, Reporting, and mining.
	\item By using end-user access tools users can link with the warehouse\cite{Sharma2021}.
	
\end{itemize}

\subsection{Datalake}
A data lake is a scalable storage and analysis system for data of any type, retained in their native format and used
mainly by data specialists (statisticians, data scientists or analysts) for knowledge extraction. Its characteristics include\cite{sawadogo2021data}:

\begin{itemize}	
	
	\item a metadata catalog that enforces data quality.
	\item data governance policies and tools.
	\item accessibility to various kinds of users.
	\item integration of any type of data.
	\item a logical and physical organization.
	\item scalability in terms of storage and processing.
	
\end{itemize}

\subsection{Datalake Components}

\subsubsection{Data Ingestion}
Temporary loading layer in which data passes through basic checks before being stored in the raw data layer. It can
perform\cite{Dertiano2021}:

\begin{itemize}	
	
	\item Basic quality controls, such as possible filters according to the origin of the data, discarding unknown sources.
	\item Data encryption processes if required for security reasons.
	\item Simple metadata and traceability records by tags, storing the origin of the data, date and time of loading, format and
	other technical characteristics, privacy and security level, encryption algorithm, etc.

\end{itemize}

\subsubsection{Data Storage}
Layer without established schema where all data, structured or unstructured, are stored without undergoing adaptations. It is a
layer that requires expert data discovery analysts using big data tools (Hive, Spark, Map Reduce, etc.)\cite{Dertiano2021}.

\subsubsection{Data Processing}
Once the data analysts have performed data discovery on the raw data, it may be necessary to process and adapt certain
datasets to accommodate them in a layer of recurrent use. Advanced data quality, integrity and other adaptations can take
place in this layer to provide a trusted layer of data exploration that can be accessed by other users\cite{Dertiano2021}.

\subsubsection{Data Access}
This is a more advanced layer where, finally, data is made available to business analysts. These analysts will be able to generate
reports and analysis to answer business questions and support decision making\cite{Dertiano2021}.

\subsection{Datalake Examples}

\subsubsection{Victoria University}
Victoria University, an Australian public university with more than 40,000 students, is using Cazena to collect and manage
the massive amounts of data generated by student interactions and their business systems. With Cazena’s Instant Data Lake
with Cloudera on AWS, VU has improved educational outcomes and business operations, helping to attract and retain
students\cite{Victoria2021}.

\begin{center}
	\includegraphics[width=7cm]{./images/dl-example}
\end{center}

%----------------------------------------------------------------------------------------
%	Conclusions
%----------------------------------------------------------------------------------------

\section{Conclusions}
\begin{itemize}	
	
	\item Lorem ipsum dolor sit amet, consectetur adipiscing elit. Morbi vulputate tempus molestie.
	\item Lorem ipsum dolor sit amet, consectetur adipiscing elit. Morbi vulputate tempus molestie.

\end{itemize}

%----------------------------------------------------------------------------------------
%	References
%----------------------------------------------------------------------------------------

\bibliographystyle{plain} 
\bibliography{references} 
\end{document}